\documentclass[a4paper]{article}

\usepackage[utf8]{inputenc}
\usepackage{amsmath, amsfonts, amssymb}
\usepackage{graphicx}
\usepackage{multicol}
\usepackage{xcolor}
\usepackage{enumitem}
\usepackage[compact]{titlesec}
\usepackage{setspace}
\usepackage[a4paper, margin=0cm]{geometry}

% Shrink font aggressively
\renewcommand\normalsize{\tiny}
\normalsize
\setstretch{0.75}

% Remove all vertical spacing everywhere
\setlength{\parskip}{0pt}
\setlength{\parindent}{0pt}
\setlist{noitemsep, topsep=0pt, partopsep=0pt, parsep=0pt, itemsep=0pt}

% Section formatting tiny
\titleformat*{\section}{\tiny\bfseries}
\titlespacing*{\section}{0pt}{0.3ex}{0.2ex}

% Column spacing minimal
\setlength{\columnsep}{0.25cm}

% Red bold answers
\newcommand{\answer}[1]{\textcolor{red}{\textbf{#1}}}

\begin{document}
\pagestyle{empty}

\begin{multicols}{2}

\section*{Quiz 1: Estimation and Sampling Distributions}
\begin{enumerate}
    \item A parameter is an estimate of a statistic. True / \answer{False}
    \item The sample mean of a Normal population with mean $\mu$ and variance $\sigma^2$ has which distribution?
    \begin{enumerate}
        \item \answer{$N(\mu, \sigma^2/n)$}
        \item $t$-distribution
        \item $N(\mu, \sigma^2)$
        \item Uniform distribution
    \end{enumerate}
    \item Which of the following properties of a CDF are TRUE?
    \begin{enumerate}
        \item \answer{$F(x)$ is non-decreasing}
        \item \answer{$\lim_{x\to-\infty} F(x) = 0$ and $\lim_{x\to+\infty} F(x) = 1$}
        \item \answer{$0 \le F(x) \le 1$}
        \item $F(x)$ is always continuous
    \end{enumerate}
    \item The 95\% confidence interval for estimating $\mu$ is:
    \begin{enumerate}
        \item $\bar{X} \pm 1.96 \frac{\sigma^2}{n}$
        \item $\bar{X} \pm 1.96 \sigma$
        \item $\bar{X} \pm 1.96 \sigma^2$
        \item \answer{$\bar{X} \pm 1.96 \frac{\sigma}{\sqrt{n}}$}
    \end{enumerate}
    \item The Empirical CDF is a continuous and smooth function. True / \answer{False}
    \item Convergence in probability is stronger than convergence in distribution. \answer{True} / False
    \item Which statement about the method of moments estimator is NOT TRUE?
    \begin{enumerate}
        \item $\hat{\theta}$ exists with probability 1 under regularity conditions.
        \item The estimator $\hat{\theta}$ is always unbiased for finite $n$.
        \item \answer{$\hat{\theta}$ converges in probability to $\theta^*$ as $n \to \infty$.}
        \item $\sqrt{n}(\hat{\theta} - \theta^*)$ follows exactly a normal distribution for any $n$.
    \end{enumerate}
    \item Kullback-Leibler divergence is a symmetric distance measure. True / \answer{False}
    \item The MLE $\hat{\theta}$ is obtained by maximizing:
    \begin{enumerate}
        \item The KL divergence
        \item The prior distribution
        \item \answer{The log-likelihood function}
        \item The empirical CDF
    \end{enumerate}
    \item For $X_i \sim \text{Unif}(0, \theta)$, the MLE for $\theta$ is:
    \begin{enumerate}
        \item $\min(X_i)$
        \item $\bar{X}$
        \item $\sum X_i$
        \item \answer{$\max(X_i)$}
    \end{enumerate}
\end{enumerate}

\section*{Quiz 2: MLE and Fisher Information}
\begin{enumerate}
    \item Which are properties of the Maximum Likelihood Estimator (MLE)?
    \begin{enumerate}
        \item \answer{Equivariance}
        \item \answer{Asymptotic normality}
        \item \answer{Asymptotic efficiency}
        \item \answer{Consistency}
    \end{enumerate}
    \item The score function $s(X; \theta)$ measures:
    \begin{enumerate}
        \item \answer{Sensitivity of the log-likelihood to $\theta$}
        \item Sample variance of $X$
        \item Bias of the estimator
        \item Expected value of $X$
    \end{enumerate}
    \item For i.i.d. data, Fisher information for $n$ samples is:
    \begin{enumerate}
        \item $I_n(\theta) = n^2 I(\theta)$
        \item \answer{$I_n(\theta) = n I(\theta)$}
        \item $I_n(\theta) = I(\theta)^n$
        \item $I_n(\theta) = I(\theta)n$
    \end{enumerate}
    \item The Fisher information is:
    \begin{enumerate}
        \item \answer{$I(\theta) = \text{Var}_{\theta}[s(X; \theta)]$}
        \item $I(\theta) = E_{\theta}[s(X; \theta)]$
        \item $I(\theta) = \text{Cov}_{\theta}[X, s(X; \theta)]$
        \item \answer{$I(\theta) = E_{\theta}[s(X; \theta)^2]$}
    \end{enumerate}
    \item The Cramér-Rao bound describes this best:
    \begin{enumerate}
        \item Holds only for biased estimators
        \item \answer{No unbiased estimator can have smaller variance}
        \item The variance equals $1/I_n(\theta)$
        \item Applies only to Normal data
    \end{enumerate}
    \item The Delta Method approximates:
    \begin{enumerate}
        \item Likelihood function
        \item \answer{Sampling distribution of a function of an estimator}
        \item Confidence interval
        \item Fisher information
    \end{enumerate}
    \item For $X_i \sim \text{Bernoulli}(p)$, Fisher information per observation is:
    \begin{enumerate}
        \item $\frac{1}{n(p(1-p))}$
        \item \answer{$\frac{1}{p(1-p)}$}
        \item $np(1-p)$
        \item $p(1-p)$
    \end{enumerate}
    \item The Wald confidence interval for $\theta$ is:
    \begin{enumerate}
        \item \answer{$(\hat{\theta} - z_{\alpha/2}se(\hat{\theta}), \hat{\theta} + z_{\alpha/2}se(\hat{\theta}))$}
        \item $(z_{\alpha/2}/\hat{\theta}, z_{\alpha/2}\hat{\theta})$
        \item $(\hat{\theta} - z_{\alpha/2}, \hat{\theta} + z_{\alpha/2})$
        \item $(\hat{\theta} \pm se(\hat{\theta})^2)$
    \end{enumerate}
    \item The asymptotic relative efficiency of $U_n$ vs. $T_n$ is:
    \begin{enumerate}
        \item $\pi/2$
        \item $1$
        \item \answer{$2/\pi$}
        \item $0.63$
    \end{enumerate}
    \item In bootstrap variance estimation, data are obtained by:
    \begin{enumerate}
        \item \answer{Resampling with replacement}
        \item Generating new i.i.d. samples from F
        \item Resampling without replacement
        \item Randomizing indices without replacement
    \end{enumerate}
\end{enumerate}

\section*{Quiz 3: Regression and Nonparametric Estimation}
\begin{enumerate}
    \item In simple linear regression, which are true about $\varepsilon_i$?
    \begin{enumerate}
        \item \answer{Zero mean conditional on X}
        \item \answer{i.i.d.}
        \item \answer{Constant variance conditional on X}
    \end{enumerate}
    \item OLS estimates $\beta_0$ and $\beta_1$ by minimizing RSS. \answer{True} / False
    \item A 95\% prediction interval differs from a 95\% CI because:
    \begin{enumerate}
        \item It ignores parameter uncertainty
        \item They are identical at sample mean
        \item \answer{It includes both parameter uncertainty and observation noise}
        \item Same width for all x
    \end{enumerate}
    \item In logistic regression, the decision boundary is:
    \begin{enumerate}
        \item $\sigma(s) = 1$
        \item $f(X, \beta) = 1$
        \item \answer{$\sigma(s) = 0.5$, i.e. $f(X, \beta) = 0$}
        \item Depends on scale of X
    \end{enumerate}
    \item The plot shows how estimated error changes with bandwidth h:
    \begin{enumerate}
        \item \answer{Very small $h$ gives high variance}
        \item All of the above
        \item None of the above
        \item \answer{Very large $h$ increases bias}
    \end{enumerate}
    \item Reducing $h$ always improves accuracy because it reduces bias. True / \answer{False}
    \item Which is not required for a kernel $K(x)$?
    \begin{enumerate}
        \item \answer{Must be Gaussian}
        \item $\int K(x)dx = 1$
        \item $K(x) \ge 0$
        \item $\int xK(x)dx = 0$
    \end{enumerate}
    \item In Nadaraya-Watson estimator, closer points get:
    \begin{enumerate}
        \item \answer{Higher weights}
        \item Same weight
        \item Lower weights
        \item Negative weights
    \end{enumerate}
    \item For histogram with $h^* \propto n^{-1/3}$, MISE convergence rate is:
    \begin{enumerate}
        \item \answer{$O(n^{-2/3})$}
        \item $O(n^{-1/2})$
        \item $O(n^{-1})$
        \item $O(n^{-4/5})$
    \end{enumerate}
    \item What does the shaded blue area in the plot represent?
    \begin{enumerate}
        \item Bandwidth selection curve
        \item \answer{Range of possible r(x) values at given confidence}
        \item Bias of estimator
        \item Variance of errors
    \end{enumerate}
\end{enumerate}

\columnbreak

\section*{Quiz 4: Hypothesis Testing}
\begin{enumerate}
    \item In hypothesis testing, $H_0$ usually represents:
    \begin{enumerate}
        \item The claim that parameter $\in \Theta_1$
        \item \answer{No effect or no difference}
        \item Parameter value that always changes
        \item Claim of observed changes
    \end{enumerate}
    \item A hypothesis $H_0: \theta = \theta_0$ is:
    \begin{enumerate}
        \item Composite
        \item Impossible to test
        \item Nonparametric
        \item \answer{Simple}
    \end{enumerate}
    \item The critical region of a test is:
    \begin{enumerate}
        \item All parameter values
        \item Region of Type II errors
        \item \answer{Outcomes for which we reject $H_0$}
        \item Outcomes for which we fail to reject $H_0$
    \end{enumerate}
    \item The significance level $\alpha$ of a test is:
    \begin{enumerate}
        \item \answer{Probability of Type I error}
        \item \answer{Probability of rejecting $H_0$ when true}
        \item \answer{Upper bound on $P(X_n \in \chi_1 | H_0)$}
        \item All of the above
    \end{enumerate}
    \item Match decisions with errors:
    \begin{enumerate}
        \item \answer{Fail to reject $H_0$ when $H_1$ true $\to$ Type II error}
        \item \answer{Reject $H_0$ when $H_0$ true $\to$ Type I error}
        \item Reject $H_0$ when $H_1$ true $\to$ Type II error
        \item \answer{Fail to reject $H_0$ when $H_0$ true $\to$ correct decision}
    \end{enumerate}
    \item The power of a test against $F_1$ is:
    \begin{enumerate}
        \item \answer{$W(F_1) = P(\text{reject } H_0 | F_1)$}
        \item Type II error probability
        \item Probability of failing to reject under $H_1$
        \item Both (a) and (b)
    \end{enumerate}
    \item A test is unbiased if:
    \begin{enumerate}
        \item Rejects only when $p < 0.05$
        \item Power $\to 1$ as $n \to \infty$
        \item Smallest Type I error
        \item \answer{$W(F_1) \ge W(F_0)$ for all $F_1, F_0$}
    \end{enumerate}
    \item A test is most powerful at level $\alpha$ if:
    \begin{enumerate}
        \item Smallest possible Type I error
        \item \answer{Maximizes $P(\text{reject } H_0 | H_1)$}
        \item \answer{Minimizes $1 - W(F_1)$ among same-level tests}
        \item Both (b) and (c)
    \end{enumerate}
    \item A p-value can be interpreted as:
    \begin{enumerate}
        \item Probability $H_0$ is true
        \item \answer{Smallest $\alpha$ where $H_0$ rejected}
        \item \answer{Probability (under $H_0$) of observing test statistic as extreme}
        \item \answer{Measure of evidence against $H_0$}
    \end{enumerate}
    \item When testing $H_0: \mu = \mu_0$ for Normal data with known $\sigma^2$, T follows:
    \begin{enumerate}
        \item $t$-distribution with $n-1$ d.f.
        \item \answer{Standard normal}
        \item Chi-square
        \item Uniform
    \end{enumerate}
\end{enumerate}

\section*{Quiz 5: Hypothesis Testing II}
\begin{enumerate}
    \item The Wald test statistic asymptotically follows which distribution under $H_0$?
    \begin{enumerate}
        \item Uniform distribution
        \item $\chi^2$ distribution
        \item \answer{N(0,1)}
        \item $t$-distribution
    \end{enumerate}
    \item The Wald test rejects $H_0$ iff $\theta_0$ lies outside the $(1-\alpha)\%$ CI.
    \begin{enumerate}
        \item \answer{True}
        \item False
    \end{enumerate}
    \item Pearson's chi-squared test statistic asymptotically follows:
    \begin{enumerate}
        \item $\chi^2_{k-2}$
        \item N(0,1)
        \item $\chi^2_k$
        \item \answer{$\chi^2_{k-1}$}
    \end{enumerate}
    \item Which are properties of the chi-square distribution $\chi^2_k$?
    \begin{enumerate}
        \item \answer{$E(\chi^2_k) = k$}
        \item \answer{If $Z_i \sim N(0,1)$, then $\sum Z_i^2 \sim \chi^2_k$}
        \item $\chi^2_k$ is symmetric around 0
        \item \answer{$\text{Var}(\chi^2_k) = 2k$}
    \end{enumerate}
    \item The likelihood-ratio test statistic is defined as:
    \begin{enumerate}
        \item $\lambda_n = L_n(\hat{\theta}) / L_n(\theta_0)$
        \item $\lambda_n = 2 L_n(\hat{\theta}) / L_n(\theta_0)$
        \item $\lambda_n = \log (L_n(\hat{\theta}) / L_n(\theta_0))$
        \item \answer{$\lambda_n = 2 \log (L_n(\hat{\theta}) / L_n(\theta_0))$}
    \end{enumerate}
    \item According to Wilks' theorem, the LRT statistic asymptotically follows:
    \begin{enumerate}
        \item $\chi^2_q$
        \item \answer{$\chi^2_{r-q}$}
        \item $\chi^2_r$
        \item N(0,1)
    \end{enumerate}
    \item The permutation test requires asymptotic normality.
    \begin{enumerate}
        \item True
        \item \answer{False}
    \end{enumerate}
    \item In the permutation test, under $H_0$, all permutations are:
    \begin{enumerate}
        \item Following a $\chi^2$ distribution
        \item \answer{Uniformly distributed (equally likely)}
        \item Following a $t$-distribution
        \item Normally distributed
    \end{enumerate}
    \item When testing $m$ independent hypotheses, what happens to FWER?
    \begin{enumerate}
        \item \answer{FWER increases and approaches 1}
        \item FWER stays constant at 0.05
        \item FWER decreases
        \item FWER remains undefined
    \end{enumerate}
    \item The Bonferroni correction controls the FWER by rejecting $H_{0i}$ when:
    \begin{enumerate}
        \item $P_i < \alpha$
        \item $P_i < \alpha \times m$
        \item \answer{$P_i < \alpha/m$}
        \item $P_i < \sqrt{\alpha/m}$
    \end{enumerate}
\end{enumerate}

\end{multicols}
\end{document}